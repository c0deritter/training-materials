\subsection{Preface}

\begin{frame}
\frametitle{Preconditions}
  \begin{itemize}
     \item You should have basic knowledge about Yocto, bitbake and it's concepts
     \item If anything might be unclear during the presentation, don't hesitate to ask
     \item If anything is to boring, please tell me so we can skip forward
     \item Most important part is your contribution. especially during the practical part
  \end{itemize}
\end{frame}

\begin{frame}
\frametitle{What are we talking about?}
  \begin{itemize}
     \item You're project is pretty close to Hardware.
     \item You decided to use yocto as base for your work.
     \item Where to put all the stuff you developed?
     \item Time flies and during the project you need to modify stuff that usually comes from poky or open embedded itself
  \end{itemize}
\end{frame}

\begin{frame}
\frametitle{Basic rules}
  \begin{itemize}
     \item All modifications are made in \textbf{your} {\em meta-<project-bsp>} layer.
     \item Editing Poky is a \textbf{no-go}!
     \item \code{build/conf/local.conf} should be your scratchboard only
     \item Distinguish between \code{distro}, \code{machine} and \code{image}
     \item Remember: conf is global, recipe is local
     \item Recommendation: separate BSP-, Distro- and Application- Layer
        \begin{itemize}
           \item {\em for smaller projects (most often) you can at least combine BSP- and Distor- Layer}
        \end{itemize}
  \end{itemize}
\end{frame}


\ifdefempty{\trainer}{}{
  \input{../slides/first-slides/\trainer.tex}
}

\begin{frame}
\frametitle{Rights to copy}
   \scriptsize
   © Copyright 2004-\the\year, Bootlin \\
   © Copyright      \the\year, Coderitter \\
   {\bf License: Creative Commons Attribution - Share Alike 3.0} \\
   \url{https://creativecommons.org/licenses/by-sa/3.0/legalcode} \\

   You are free:
   \begin{itemize}
     \item to copy, distribute, display, and perform the work
     \item to make derivative works
     \item to make commercial use of the work
   \end{itemize}

   Under the following conditions:
   \begin{itemize}
     \item {\bf Attribution}. You must give the original author credit.
     \item {\bf Share Alike}. If you alter, transform, or build upon
           this work, you may distribute the resulting work only under
           a license identical to this one.
     \item For any reuse or distribution, you must make clear to others
           the license terms of this work.
     \item Any of these conditions can be waived if you get permission
           from the copyright holder.
    \end{itemize}

    Your fair use and other rights are in no way affected by the above.
    \vfill
    {\bf Document sources:}
    \url{github.com:c0deritter/training-materials.git} \\
    {\bf Document sources (original):}
    \url{https://github.com/bootlin/training-materials/} \\
\end{frame}

%% If the materials a generated for a real session, not for the website

\ifdefempty{\trainer}{}{
  \begin{frame}
  \frametitle{Electronic copies of these documents}
     \begin{itemize}
        \item Electronic copies of your particular version of the
              materials are available on:\\
              {\scriptsize \url{\sessionurl}} \\
        \item Open the corresponding documents and use them throughout
              the course to look for explanations given earlier by the
              instructor.
        \item You will need these electronic versions because
	      we neither print any index nor any table of contents
	      (quite high environmental cost for little added value)
        \item For future reference, see the first slide to see where
              document updates will be available.
    \end{itemize}
  \end{frame}
}

\begin{frame}
\frametitle{Hyperlinks in the document}
  There are many hyperlinks in the document
  \begin{itemize}
    \item Regular hyperlinks:\\
          \url{https://kernel.org/}
    \item Kernel documentation links:\\
	  \kerneldochtml{dev-tools/kasan}
    \item Links to kernel source files and directories:\\
	  \kdir{drivers/input} \\
	  \kfile{include/linux/fb.h}
    \item Links to the declarations, definitions and instances
          of kernel symbols (functions, types, data, structures): \\
	  \kfunc{platform_get_irq} \\
	  \ksym{GFP_KERNEL} \\
	  \kstruct{file_operations}
  \end{itemize}
\end{frame}

